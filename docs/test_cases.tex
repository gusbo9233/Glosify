\documentclass[11pt,a4paper]{article}
\usepackage[utf8]{inputenc}
\usepackage[T1]{fontenc}
\usepackage{geometry}
\usepackage{hyperref}
\usepackage{enumitem}
\usepackage{booktabs}
\usepackage{longtable}
\usepackage{xcolor}
\usepackage{tabularx}

% Page setup
\geometry{margin=2.5cm}
\hypersetup{
    colorlinks=true,
    linkcolor=blue,
    filecolor=magenta,      
    urlcolor=cyan,
    pdftitle={Test Cases Specification},
}

% Custom commands
\newcommand{\testid}[1]{\textbf{TC-#1}}
\newcommand{\priority}[1]{\textit{Priority: #1}}
\newcommand{\status}[1]{\textit{Status: #1}}
\newcommand{\reqref}[1]{\textit{Requirement: REQ-#1}}

\title{Test Cases Specification}
\author{Test Team}
\date{\today}

\begin{document}

\maketitle

\begin{abstract}
This document specifies the test cases for the system. It includes functional test cases, integration test cases, system test cases, and acceptance test cases. Each test case includes detailed steps, expected results, and traceability to requirements.
\end{abstract}

\tableofcontents
\newpage

\section{Introduction}

\subsection{Purpose}
This document describes the test cases for verifying that the system meets its specified requirements. It serves as a reference for testers, developers, and quality assurance personnel.

\subsection{Scope}
% Describe the scope of testing here
The test cases cover...

\subsection{Test Environment}
% Describe the test environment
\begin{itemize}
    \item \textbf{Hardware:} Test hardware specifications
    \item \textbf{Software:} Test software versions and configurations
    \item \textbf{Test Data:} Description of test data used
    \item \textbf{Test Tools:} Testing tools and frameworks used
\end{itemize}

\subsection{Test Strategy}
% Describe overall test strategy
\begin{itemize}
    \item Unit testing approach
    \item Integration testing approach
    \item System testing approach
    \item Acceptance testing approach
\end{itemize}

\subsection{Definitions, Acronyms, and Abbreviations}
\begin{itemize}
    \item \textbf{TC}: Test Case
    \item \textbf{SUT}: System Under Test
    \item \textbf{Test Term}: Definition
\end{itemize}

\subsection{References}
\begin{itemize}
    \item Requirements Specification
    \item Design Document
    \item Other relevant documents
\end{itemize}

\section{Test Case Template}

Each test case follows this structure:
\begin{itemize}
    \item \textbf{Test Case ID}: Unique identifier
    \item \textbf{Test Case Name}: Descriptive name
    \item \textbf{Requirement Reference}: Related requirement(s)
    \item \textbf{Priority}: High, Medium, or Low
    \item \textbf{Test Type}: Functional, Integration, System, Acceptance, etc.
    \item \textbf{Preconditions}: Conditions that must be met before execution
    \item \textbf{Test Steps}: Detailed step-by-step instructions
    \item \textbf{Test Data}: Input data required
    \item \textbf{Expected Results}: What should happen
    \item \textbf{Actual Results}: What actually happened (filled during execution)
    \item \textbf{Status}: Pass, Fail, Blocked, or Not Executed
    \item \textbf{Notes}: Additional comments or observations
\end{itemize}

\section{Functional Test Cases}

\subsection{Feature Category 1}
% Example: User Management, Authentication, etc.

\subsubsection{TC-001: Test Case Title}
\testid{001} \priority{High} \status{Not Executed} \reqref{001}

\textbf{Test Case Name:}
% Descriptive name of the test case

\textbf{Test Type:}
Functional Test

\textbf{Requirement Reference:}
REQ-001

\textbf{Objective:}
% What this test case is trying to verify

\textbf{Preconditions:}
\begin{itemize}
    \item Precondition 1
    \item Precondition 2
\end{itemize}

\textbf{Test Steps:}
\begin{enumerate}
    \item Step 1: Description of action
    \item Step 2: Description of action
    \item Step 3: Description of action
\end{enumerate}

\textbf{Test Data:}
\begin{itemize}
    \item Input 1: Value
    \item Input 2: Value
\end{itemize}

\textbf{Expected Results:}
\begin{itemize}
    \item Expected outcome 1
    \item Expected outcome 2
    \item Expected outcome 3
\end{itemize}

\textbf{Actual Results:}
% To be filled during test execution
\begin{itemize}
    \item 
\end{itemize}

\textbf{Status:}
Not Executed

\textbf{Notes:}
% Additional comments

\subsubsection{TC-002: Test Case Title}
\testid{002} \priority{High} \status{Not Executed} \reqref{002}

\textbf{Test Case Name:}
% Descriptive name

\textbf{Test Type:}
Functional Test

\textbf{Requirement Reference:}
REQ-002

\textbf{Objective:}
% What this test case is trying to verify

\textbf{Preconditions:}
\begin{itemize}
    \item Precondition 1
\end{itemize}

\textbf{Test Steps:}
\begin{enumerate}
    \item Step 1: Description
    \item Step 2: Description
\end{enumerate}

\textbf{Test Data:}
\begin{itemize}
    \item Input 1: Value
\end{itemize}

\textbf{Expected Results:}
\begin{itemize}
    \item Expected outcome 1
    \item Expected outcome 2
\end{itemize}

\textbf{Actual Results:}
% To be filled during test execution

\textbf{Status:}
Not Executed

\textbf{Notes:}

% Add more functional test cases as needed

\subsection{Feature Category 2}
% Another feature category

\subsubsection{TC-003: Test Case Title}
\testid{003} \priority{Medium} \status{Not Executed} \reqref{003}

\textbf{Test Case Name:}
% Descriptive name

\textbf{Test Type:}
Functional Test

\textbf{Requirement Reference:}
REQ-003

\textbf{Objective:}
% What this test case is trying to verify

\textbf{Preconditions:}
\begin{itemize}
    \item Precondition 1
\end{itemize}

\textbf{Test Steps:}
\begin{enumerate}
    \item Step 1: Description
    \item Step 2: Description
\end{enumerate}

\textbf{Test Data:}
\begin{itemize}
    \item Input 1: Value
\end{itemize}

\textbf{Expected Results:}
\begin{itemize}
    \item Expected outcome 1
\end{itemize}

\textbf{Actual Results:}
% To be filled during test execution

\textbf{Status:}
Not Executed

\textbf{Notes:}

\section{Integration Test Cases}

\subsection{Integration Scenario 1}

\subsubsection{TC-INT-001: Integration Test Case Title}
\testid{INT-001} \priority{High} \status{Not Executed}

\textbf{Test Case Name:}
% Descriptive name

\textbf{Test Type:}
Integration Test

\textbf{Requirement Reference:}
REQ-XXX, REQ-YYY

\textbf{Objective:}
% What integration is being tested

\textbf{Components Under Test:}
\begin{itemize}
    \item Component 1
    \item Component 2
\end{itemize}

\textbf{Preconditions:}
\begin{itemize}
    \item Precondition 1
    \item Precondition 2
\end{itemize}

\textbf{Test Steps:}
\begin{enumerate}
    \item Step 1: Description
    \item Step 2: Description
    \item Step 3: Description
\end{enumerate}

\textbf{Test Data:}
\begin{itemize}
    \item Input 1: Value
\end{itemize}

\textbf{Expected Results:}
\begin{itemize}
    \item Expected outcome 1
    \item Expected outcome 2
\end{itemize}

\textbf{Actual Results:}
% To be filled during test execution

\textbf{Status:}
Not Executed

\textbf{Notes:}

\section{System Test Cases}

\subsection{System Test Scenario 1}

\subsubsection{TC-SYS-001: System Test Case Title}
\testid{SYS-001} \priority{High} \status{Not Executed}

\textbf{Test Case Name:}
% Descriptive name

\textbf{Test Type:}
System Test

\textbf{Requirement Reference:}
REQ-XXX

\textbf{Objective:}
% What system behavior is being tested

\textbf{Preconditions:}
\begin{itemize}
    \item Precondition 1
\end{itemize}

\textbf{Test Steps:}
\begin{enumerate}
    \item Step 1: Description
    \item Step 2: Description
\end{enumerate}

\textbf{Test Data:}
\begin{itemize}
    \item Input 1: Value
\end{itemize}

\textbf{Expected Results:}
\begin{itemize}
    \item Expected outcome 1
\end{itemize}

\textbf{Actual Results:}
% To be filled during test execution

\textbf{Status:}
Not Executed

\textbf{Notes:}

\section{Performance Test Cases}

\subsection{Performance Test Scenario 1}

\subsubsection{TC-PERF-001: Performance Test Case Title}
\testid{PERF-001} \priority{High} \status{Not Executed}

\textbf{Test Case Name:}
% Descriptive name

\textbf{Test Type:}
Performance Test

\textbf{Requirement Reference:}
REQ-NFR-001

\textbf{Objective:}
% What performance aspect is being tested

\textbf{Performance Criteria:}
\begin{itemize}
    \item Response time: Must be less than X seconds
    \item Throughput: Must handle Y requests per second
    \item Resource usage: CPU usage must be below Z\%
\end{itemize}

\textbf{Test Environment:}
\begin{itemize}
    \item Hardware configuration
    \item Network conditions
    \item Load conditions
\end{itemize}

\textbf{Preconditions:}
\begin{itemize}
    \item Precondition 1
\end{itemize}

\textbf{Test Steps:}
\begin{enumerate}
    \item Step 1: Description
    \item Step 2: Description
    \item Step 3: Description
\end{enumerate}

\textbf{Test Data:}
\begin{itemize}
    \item Load profile
    \item Number of concurrent users
    \item Data volume
\end{itemize}

\textbf{Expected Results:}
\begin{itemize}
    \item Response time: Less than X seconds
    \item Throughput: Y requests per second
    \item No errors or failures
\end{itemize}

\textbf{Actual Results:}
% To be filled during test execution
\begin{itemize}
    \item Response time: 
    \item Throughput: 
    \item Errors/Failures: 
\end{itemize}

\textbf{Status:}
Not Executed

\textbf{Notes:}

\section{Security Test Cases}

\subsection{Security Test Scenario 1}

\subsubsection{TC-SEC-001: Security Test Case Title}
\testid{SEC-001} \priority{High} \status{Not Executed}

\textbf{Test Case Name:}
% Descriptive name

\textbf{Test Type:}
Security Test

\textbf{Requirement Reference:}
REQ-NFR-002

\textbf{Objective:}
% What security aspect is being tested

\textbf{Preconditions:}
\begin{itemize}
    \item Precondition 1
\end{itemize}

\textbf{Test Steps:}
\begin{enumerate}
    \item Step 1: Description
    \item Step 2: Description
\end{enumerate}

\textbf{Test Data:}
\begin{itemize}
    \item Input 1: Value (e.g., malicious input)
\end{itemize}

\textbf{Expected Results:}
\begin{itemize}
    \item System should reject unauthorized access
    \item System should handle input securely
    \item No sensitive data should be exposed
\end{itemize}

\textbf{Actual Results:}
% To be filled during test execution

\textbf{Status:}
Not Executed

\textbf{Notes:}

\section{Usability Test Cases}

\subsection{Usability Test Scenario 1}

\subsubsection{TC-USA-001: Usability Test Case Title}
\testid{USA-001} \priority{Medium} \status{Not Executed}

\textbf{Test Case Name:}
% Descriptive name

\textbf{Test Type:}
Usability Test

\textbf{Requirement Reference:}
REQ-NFR-004

\textbf{Objective:}
% What usability aspect is being tested

\textbf{User Profile:}
% Type of user for this test

\textbf{Preconditions:}
\begin{itemize}
    \item Precondition 1
\end{itemize}

\textbf{Test Steps:}
\begin{enumerate}
    \item Step 1: Description
    \item Step 2: Description
\end{enumerate}

\textbf{Test Data:}
\begin{itemize}
    \item Input 1: Value
\end{itemize}

\textbf{Expected Results:}
\begin{itemize}
    \item User can complete task within X minutes
    \item Interface is intuitive and clear
    \item Error messages are helpful
\end{itemize}

\textbf{Actual Results:}
% To be filled during test execution

\textbf{Status:}
Not Executed

\textbf{Notes:}

\section{Acceptance Test Cases}

\subsection{Acceptance Test Scenario 1}

\subsubsection{TC-ACC-001: Acceptance Test Case Title}
\testid{ACC-001} \priority{High} \status{Not Executed}

\textbf{Test Case Name:}
% Descriptive name

\textbf{Test Type:}
Acceptance Test

\textbf{Requirement Reference:}
REQ-XXX

\textbf{Objective:}
% What user acceptance criteria is being tested

\textbf{User Story:}
% Related user story if applicable

\textbf{Preconditions:}
\begin{itemize}
    \item Precondition 1
\end{itemize}

\textbf{Test Steps:}
\begin{enumerate}
    \item Step 1: Description
    \item Step 2: Description
\end{enumerate}

\textbf{Test Data:}
\begin{itemize}
    \item Input 1: Value
\end{itemize}

\textbf{Expected Results:}
\begin{itemize}
    \item Expected outcome 1
\end{itemize}

\textbf{Actual Results:}
% To be filled during test execution

\textbf{Status:}
Not Executed

\textbf{Notes:}

\section{Test Execution Summary}

\subsection{Test Execution Statistics}
% Summary table of test execution

\begin{longtable}{|p{2cm}|p{2cm}|p{2cm}|p{2cm}|p{2cm}|p{2cm}|}
\hline
\textbf{Test Type} & \textbf{Total} & \textbf{Passed} & \textbf{Failed} & \textbf{Blocked} & \textbf{Not Executed} \\
\hline
Functional & 0 & 0 & 0 & 0 & 0 \\
\hline
Integration & 0 & 0 & 0 & 0 & 0 \\
\hline
System & 0 & 0 & 0 & 0 & 0 \\
\hline
Performance & 0 & 0 & 0 & 0 & 0 \\
\hline
Security & 0 & 0 & 0 & 0 & 0 \\
\hline
Usability & 0 & 0 & 0 & 0 & 0 \\
\hline
Acceptance & 0 & 0 & 0 & 0 & 0 \\
\hline
\textbf{Total} & \textbf{0} & \textbf{0} & \textbf{0} & \textbf{0} & \textbf{0} \\
\hline
\end{longtable}

\subsection{Test Coverage}
% Coverage analysis

\textbf{Requirements Coverage:}
\begin{itemize}
    \item Total Requirements: X
    \item Requirements with Test Cases: Y
    \item Coverage: Z\%
\end{itemize}

\section{Test Traceability Matrix}

% Matrix showing which test cases verify which requirements

\begin{longtable}{|p{2cm}|p{8cm}|p{3cm}|}
\hline
\textbf{Requirement ID} & \textbf{Test Case IDs} & \textbf{Coverage Status} \\
\hline
REQ-001 & TC-001, TC-002 & Covered \\
\hline
REQ-002 & TC-003 & Covered \\
\hline
REQ-NFR-001 & TC-PERF-001 & Covered \\
\hline
REQ-NFR-002 & TC-SEC-001 & Covered \\
\hline
% Add more rows as needed
\end{longtable}

\section{Defect Log}

% Optional: Track defects found during testing

\subsection{Defect Summary}
\begin{longtable}{|p{1.5cm}|p{2cm}|p{3cm}|p{2cm}|p{2cm}|p{2cm}|}
\hline
\textbf{Defect ID} & \textbf{Test Case ID} & \textbf{Description} & \textbf{Severity} & \textbf{Status} & \textbf{Assigned To} \\
\hline
DEF-001 & TC-001 & Defect description & High & Open & Developer \\
\hline
% Add more rows as needed
\end{longtable}

\section{Appendices}

\subsection{Appendix A: Test Data}
% Detailed test data used in test cases

\subsection{Appendix B: Test Environment Setup}
% Detailed instructions for setting up test environment

\subsection{Appendix C: Test Tools}
% List of test tools and their configurations

\subsection{Appendix D: Change Log}
% Track changes to test cases
\begin{itemize}
    \item \textbf{Date}: Change description
    \item \textbf{Date}: Change description
\end{itemize}

\end{document}

