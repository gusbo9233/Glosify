\documentclass[11pt,a4paper]{article}
\usepackage[utf8]{inputenc}
\usepackage[T1]{fontenc}
\usepackage{geometry}
\usepackage{hyperref}
\usepackage{enumitem}
\usepackage{booktabs}
\usepackage{longtable}
\usepackage{xcolor}

% Page setup
\geometry{margin=2.5cm}
\hypersetup{
    colorlinks=true,
    linkcolor=blue,
    filecolor=magenta,      
    urlcolor=cyan,
    pdftitle={Requirements Specification},
}

% Custom commands
\newcommand{\storyid}[1]{\textbf{US-#1}}
\newcommand{\priority}[1]{\textit{Priority: #1}}
\newcommand{\status}[1]{\textit{Status: #1}}
\newcommand{\storypoints}[1]{\textit{Story Points: #1}}

\title{Requirements Specification\\User Stories}
\author{Project Team}
\date{\today}

\begin{document}

\maketitle

\begin{abstract}
This document specifies the functional and non-functional requirements for the system in the form of user stories.
Each user story follows the format: "As a [user type], I want [goal], so that [benefit]".
The user stories are organized by feature/epic and include acceptance criteria, priorities, story points, and dependencies.
\end{abstract}

\tableofcontents
\newpage

\section{Introduction}

\subsection{Purpose}
This document describes the requirements for the system. It serves as a reference for developers, testers, and stakeholders throughout the development lifecycle.

\subsection{Scope}
% Describe the scope of the system here
The system scope includes...

\subsection{Definitions, Acronyms, and Abbreviations}
\begin{itemize}
    \item \textbf{Term}: Definition
    \item \textbf{Acronym}: Full form
\end{itemize}

\subsection{References}
\begin{itemize}
    \item Reference 1
    \item Reference 2
\end{itemize}

\subsection{Overview}
This document is organized as follows:
\begin{itemize}
    \item Section 2: Overall Description
    \item Section 3: User Stories (Functional Requirements)
    \item Section 4: Non-Functional Requirements
    \item Section 5: System Requirements
    \item Section 6: Interface Requirements
    \item Section 7: Constraints
\end{itemize}

\subsection{User Story Format}
Each user story follows the standard format:
\begin{itemize}
    \item \textbf{As a} [type of user]
    \item \textbf{I want} [goal/feature]
    \item \textbf{So that} [benefit/reason]
\end{itemize}

Additionally, each story includes:
\begin{itemize}
    \item Story ID (US-XXX)
    \item Priority (High, Medium, Low)
    \item Story Points (effort estimation)
    \item Status (Backlog, Planned, In Progress, Done, etc.)
    \item Acceptance Criteria
    \item Dependencies (other user stories)
    \item Notes (optional)
\end{itemize}

\section{Overall Description}

\subsection{Product Perspective}
% Describe how the system fits into the larger context

\subsection{Product Functions}
% High-level overview of main functions
\begin{itemize}
    \item Function 1
    \item Function 2
    \item Function 3
\end{itemize}

\subsection{User Classes and Characteristics}
% Describe different user types
\begin{itemize}
    \item \textbf{User Type 1}: Description
    \item \textbf{User Type 2}: Description
\end{itemize}

\subsection{Operating Environment}
% Describe where the system will operate
\begin{itemize}
    \item Operating systems
    \item Hardware requirements
    \item Network requirements
\end{itemize}

\subsection{Design and Implementation Constraints}
% Any constraints that affect design or implementation
\begin{itemize}
    \item Constraint 1
    \item Constraint 2
\end{itemize}

\subsection{Assumptions and Dependencies}
% List assumptions and dependencies
\begin{itemize}
    \item Assumption 1
    \item Dependency 1
\end{itemize}

\section{User Stories}

\subsection{Epic 1: Feature Category}
% Example: User Management, Authentication, Data Processing, etc.

\subsubsection{US-001: User Story Title}
\storyid{001} \priority{High} \storypoints{5} \status{Backlog}

\textbf{User Story:}
\begin{itemize}
    \item \textbf{As a} [type of user]
    \item \textbf{I want} [goal/feature]
    \item \textbf{So that} [benefit/reason]
\end{itemize}

\textbf{Acceptance Criteria:}
\begin{itemize}
    \item Given [context], when [action], then [expected result]
    \item Given [context], when [action], then [expected result]
    \item Given [context], when [action], then [expected result]
\end{itemize}

\textbf{Dependencies:}
% Other user stories this depends on
\begin{itemize}
    \item US-XXX
\end{itemize}

\textbf{Notes:}
% Additional context, technical notes, or considerations

\subsubsection{US-002: User Story Title}
\storyid{002} \priority{Medium} \storypoints{3} \status{Backlog}

\textbf{User Story:}
\begin{itemize}
    \item \textbf{As a} [type of user]
    \item \textbf{I want} [goal/feature]
    \item \textbf{So that} [benefit/reason]
\end{itemize}

\textbf{Acceptance Criteria:}
\begin{itemize}
    \item Given [context], when [action], then [expected result]
    \item Given [context], when [action], then [expected result]
\end{itemize}

\textbf{Dependencies:}
\begin{itemize}
    \item US-001
\end{itemize}

\textbf{Notes:}

% Add more user stories as needed

\subsection{Epic 2: Feature Category}
% Another feature category/epic

\subsubsection{US-003: User Story Title}
\storyid{003} \priority{High} \storypoints{8} \status{Backlog}

\textbf{User Story:}
\begin{itemize}
    \item \textbf{As a} [type of user]
    \item \textbf{I want} [goal/feature]
    \item \textbf{So that} [benefit/reason]
\end{itemize}

\textbf{Acceptance Criteria:}
\begin{itemize}
    \item Given [context], when [action], then [expected result]
    \item Given [context], when [action], then [expected result]
\end{itemize}

\textbf{Dependencies:}
\begin{itemize}
    \item US-XXX
\end{itemize}

\textbf{Notes:}

\section{Non-Functional Requirements}

Non-functional requirements can also be expressed as user stories or as technical requirements.

\subsection{Performance Requirements}

\subsubsection{US-NFR-001: Performance User Story}
\storyid{NFR-001} \priority{High} \storypoints{5} \status{Backlog}

\textbf{User Story:}
\begin{itemize}
    \item \textbf{As a} system user
    \item \textbf{I want} the system to respond quickly to my requests
    \item \textbf{So that} I can work efficiently without delays
\end{itemize}

\textbf{Acceptance Criteria:}
\begin{itemize}
    \item Given a normal load, when I make a request, then the system responds within X seconds
    \item Given Y concurrent users, when they all make requests, then the system handles all requests without degradation
    \item Given peak load conditions, when requests are made, then response time does not exceed Z seconds
\end{itemize}

\subsection{Security Requirements}

\subsubsection{US-NFR-002: Security User Story}
\storyid{NFR-002} \priority{High} \storypoints{8} \status{Backlog}

\textbf{User Story:}
\begin{itemize}
    \item \textbf{As a} user
    \item \textbf{I want} my data to be secure
    \item \textbf{So that} my information is protected from unauthorized access
\end{itemize}

\textbf{Acceptance Criteria:}
\begin{itemize}
    \item Given unauthorized access attempts, when someone tries to access the system, then access is denied
    \item Given sensitive data, when it is stored or transmitted, then it is encrypted
    \item Given user authentication, when credentials are entered, then they are validated securely
\end{itemize}

\subsection{Reliability Requirements}

\subsubsection{US-NFR-003: Reliability User Story}
\storyid{NFR-003} \priority{Medium} \storypoints{5} \status{Backlog}

\textbf{User Story:}
\begin{itemize}
    \item \textbf{As a} user
    \item \textbf{I want} the system to be available when I need it
    \item \textbf{So that} I can complete my work without interruption
\end{itemize}

\textbf{Acceptance Criteria:}
\begin{itemize}
    \item Given normal operations, when the system is running, then uptime is at least X\%
    \item Given system errors, when they occur, then error rate is below Y\%
    \item Given system failures, when they happen, then recovery time is within Z minutes
\end{itemize}

\subsection{Usability Requirements}

\subsubsection{US-NFR-004: Usability User Story}
\storyid{NFR-004} \priority{Medium} \storypoints{3} \status{Backlog}

\textbf{User Story:}
\begin{itemize}
    \item \textbf{As a} user
    \item \textbf{I want} the system to be easy to use
    \item \textbf{So that} I can accomplish my tasks without extensive training
\end{itemize}

\textbf{Acceptance Criteria:}
\begin{itemize}
    \item Given a new user, when they first use the system, then they can complete basic tasks within X minutes
    \item Given error situations, when they occur, then clear and helpful error messages are displayed
    \item Given common tasks, when performed, then they can be completed with minimal steps
\end{itemize}

\subsection{Maintainability Requirements}

\subsubsection{US-NFR-005: Maintainability User Story}
\storyid{NFR-005} \priority{Low} \storypoints{2} \status{Backlog}

\textbf{User Story:}
\begin{itemize}
    \item \textbf{As a} developer/maintainer
    \item \textbf{I want} the codebase to be well-structured and documented
    \item \textbf{So that} I can easily make changes and fix issues
\end{itemize}

\textbf{Acceptance Criteria:}
\begin{itemize}
    \item Given the codebase, when reviewed, then code follows established standards
    \item Given new features, when added, then documentation is updated
    \item Given bugs, when fixed, then fixes are tested and documented
\end{itemize}

\subsection{Portability Requirements}

\subsubsection{US-NFR-006: Portability User Story}
\storyid{NFR-006} \priority{Low} \storypoints{3} \status{Backlog}

\textbf{User Story:}
\begin{itemize}
    \item \textbf{As a} system administrator
    \item \textbf{I want} the system to run on different platforms
    \item \textbf{So that} I can deploy it in various environments
\end{itemize}

\textbf{Acceptance Criteria:}
\begin{itemize}
    \item Given different operating systems, when deployed, then the system runs correctly
    \item Given different hardware configurations, when installed, then the system adapts appropriately
    \item Given platform-specific features, when used, then they are abstracted appropriately
\end{itemize}

\section{System Requirements}

\subsection{Hardware Requirements}
\begin{itemize}
    \item \textbf{Minimum:}
    \begin{itemize}
        \item CPU: X GHz
        \item RAM: Y GB
        \item Storage: Z GB
    \end{itemize}
    \item \textbf{Recommended:}
    \begin{itemize}
        \item CPU: X GHz
        \item RAM: Y GB
        \item Storage: Z GB
    \end{itemize}
\end{itemize}

\subsection{Software Requirements}
\begin{itemize}
    \item Operating System: Version X or higher
    \item Runtime: Version Y or higher
    \item Database: Version Z or higher
    \item Other dependencies
\end{itemize}

\subsection{Network Requirements}
\begin{itemize}
    \item Network protocol requirements
    \item Bandwidth requirements
    \item Firewall configurations
\end{itemize}

\section{Interface Requirements}

\subsection{User Interfaces}
% Describe user interface requirements
\begin{itemize}
    \item UI Requirement 1
    \item UI Requirement 2
\end{itemize}

\subsection{Hardware Interfaces}
% Describe hardware interface requirements
\begin{itemize}
    \item Hardware Interface 1
\end{itemize}

\subsection{Software Interfaces}
% Describe software interface requirements (APIs, protocols, etc.)
\begin{itemize}
    \item API Requirement 1
    \item Protocol Requirement 1
\end{itemize}

\subsection{Communication Interfaces}
% Describe communication interface requirements
\begin{itemize}
    \item Communication Interface 1
\end{itemize}

\section{Constraints}

\subsection{Regulatory Constraints}
% Any regulatory or compliance requirements
\begin{itemize}
    \item Constraint 1
\end{itemize}

\subsection{Standards Compliance}
% Standards the system must comply with
\begin{itemize}
    \item Standard 1
    \item Standard 2
\end{itemize}

\subsection{Other Constraints}
% Any other constraints
\begin{itemize}
    \item Constraint 1
    \item Constraint 2
\end{itemize}

\section{Requirements Traceability}

% Optional: Add a traceability matrix
% This section can be used to track requirements through design, implementation, and testing

\subsection{User Story Traceability Matrix}
% Use a table to show traceability
\begin{longtable}{|p{2cm}|p{3cm}|p{3cm}|p{3cm}|p{3cm}|}
\hline
\textbf{User Story ID} & \textbf{Design Element} & \textbf{Implementation} & \textbf{Test Case} & \textbf{Status} \\
\hline
US-001 & Design-001 & Module-001 & TC-001 & Done \\
\hline
US-002 & Design-002 & Module-002 & TC-002 & In Progress \\
\hline
% Add more rows as needed
\end{longtable}

\subsection{User Story Summary}
% Summary table of user stories by status and priority

\begin{longtable}{|p{2cm}|p{2cm}|p{2cm}|p{2cm}|p{2cm}|p{2cm}|}
\hline
\textbf{Status} & \textbf{High Priority} & \textbf{Medium Priority} & \textbf{Low Priority} & \textbf{Total Stories} & \textbf{Total Points} \\
\hline
Backlog & 0 & 0 & 0 & 0 & 0 \\
\hline
Planned & 0 & 0 & 0 & 0 & 0 \\
\hline
In Progress & 0 & 0 & 0 & 0 & 0 \\
\hline
Done & 0 & 0 & 0 & 0 & 0 \\
\hline
\textbf{Total} & \textbf{0} & \textbf{0} & \textbf{0} & \textbf{0} & \textbf{0} \\
\hline
\end{longtable}

\section{Appendices}

\subsection{Appendix A: Glossary}
% Add terms and definitions here

\subsection{Appendix B: Change Log}
% Track changes to requirements
\begin{itemize}
    \item \textbf{Date}: Change description
    \item \textbf{Date}: Change description
\end{itemize}

\end{document}

